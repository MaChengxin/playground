\documentclass{beamer}

\usepackage[utf8]{inputenc}
\usepackage{hyperref}
\usepackage{mathtools}

\usetheme{Madrid}
\usecolortheme{default}

%------------------------------------------------------------
% This block of code defines the information to appear in the
% Title page.
\title[Progress of Thesis Work] %optional
{Progress of Thesis Work}

\subtitle {and discussion on current/next steps}

\author[C Ma] % (optional)
{Chengxin Ma}

\institute[TU Delft] % (optional)
{
  \inst{}
  Faculty of Electrical Engineering, Mathematics and Computer Science \\
  Delft University of Technology
}

\date{\today}

% End of title page configuration block
%------------------------------------------------------------

%------------------------------------------------------------
% The next block of commands puts the table of contents at the 
% beginning of each section and highlights the current section.

\AtBeginSection[]
{
  \begin{frame}
    \frametitle{Table of Contents}
    \tableofcontents[currentsection]
  \end{frame}
}
%------------------------------------------------------------

\begin{document}

\frame{\titlepage}

\begin{frame}
\frametitle{Table of Contents}
\tableofcontents
\end{frame}

%------------------------------------------------------------
%------------------------------------------------------------
\section{Concerns of designing parallel programs}

%------------------------------------------------------------
\begin{frame}
\frametitle{Lessons Learned from LLNL Tutorial}

\begin{itemize}
  \item Automatic vs. Manual Parallelization
  \item Understand the Problem and the Program
  \item \textbf{Partitioning}
  \item \textbf{Communication}
  \item Synchronization
  \item Data Dependencies
  \item Load Balancing
  \item \textbf{Granularity}
  \item I/O
\end{itemize}

\end{frame}

%------------------------------------------------------------
\begin{frame}
\frametitle{Automatic vs. Manual Parallelization}

Manual.
  
\end{frame}

%------------------------------------------------------------
\begin{frame}
\frametitle{Understand the Problem and the Program}

\begin{itemize}
  \item Hotspots (where most of the real work is being done)
    \begin{itemize}
      \item sorting
    \end{itemize}
  \item Bottlenecks
    \begin{itemize}
      \item Disk I/O (so we use Arrow stuff)
      \item Network/communication
       \begin{itemize}
        \item All-to-all communication
        \item But how do we quantify?
       \end{itemize}
      \item What else?
    \end{itemize}
\end{itemize}
    
\end{frame}

%------------------------------------------------------------
\begin{frame}
\frametitle{Partitioning}

\begin{itemize}
  \item \textbf{Domain decomposition} (we go this way)
  \item Functional decomposition
\end{itemize}
  
\end{frame}

%------------------------------------------------------------
\begin{frame}
\frametitle{Communication}

\begin{itemize}
  \item Communication overhead
  \item Latency and bandwidth
  \item \textbf{Visibility of communications}
    \begin{itemize}
      \item With the Message Passing Model, communications are explicit and generally quite visible and under the control of the programmer.
      \item With the Data Parallel Model \footnote{May also be referred to as the Partitioned Global Address Space (PGAS) model. 
      \url{https://computing.llnl.gov/tutorials/parallel\_comp/\#ModelsData}}, 
      communications often occur transparently to the programmer, particularly on distributed memory architectures. 
      The programmer may not even be able to know exactly how inter-task communications are being accomplished.
    \end{itemize}
  \item Synchronous vs. \textbf{asynchronous communication} (we go for asynchronous communication)
  \item Scope of communication: all-to-all
  
\end{itemize}
    
\end{frame}

%------------------------------------------------------------
\begin{frame}
\frametitle{Synchronization}

To be considered when it goes to the next phase of design.
  
\end{frame}

%------------------------------------------------------------
\begin{frame}
\frametitle{Data Dependencies}

To be considered when it goes to the next phase of design.
    
\end{frame}

%------------------------------------------------------------
\begin{frame}
\frametitle{Load Balancing}

To be considered when it goes to the next phase of design.

If data partitioning is done well, load might be almost balanced.
  
\end{frame}

%------------------------------------------------------------
\begin{frame}
\frametitle{Granularity}

Qualitative measure of the ratio of computation to communication.

\begin{itemize}
  \item Fine-grain Parallelism: repeat: sort part of the data locally, send to destination
  \item Coarse-grain Parallelism: sort all the data locally, send to destination
\end{itemize}
    
\end{frame}

%------------------------------------------------------------
\begin{frame}
\frametitle{I/O}

Use in-memory technologies instead.

Not the concern in my thesis project.
    
\end{frame}

%------------------------------------------------------------
%------------------------------------------------------------

\section{Design choices}

%------------------------------------------------------------

\begin{frame}
\frametitle{Underlying Hardware Architecture}

Nodes connected by network (referred to as Hybrid Distributed-Shared Memory in the LLNL tutorial)

Interconnection technologies:
\begin{itemize}
  \item InfiniBand
  \item 10 Gigabit Ethernet
\end{itemize}

Some \footnote{\url{https://www.hpcadvisorycouncil.com/pdf/IB\_and\_10GigE\_in\_HPC.pdf}}
says either of them is OK.
How to identify if communication bandwidth/latency is the bottleneck?

My opinion on RDMA:
from the programmer’s perspective, usage of RDMA might be implicit.
“RDMA programming interfaces are often not designed to be used by end-users directly”.
\footnote{\url{https://htor.inf.ethz.ch/blog/index.php/2016/05/15/what-are-the-real-differences-between-rdma-infiniband-rma-and-pgas/}}

\end{frame}

%------------------------------------------------------------

\begin{frame}
\frametitle{Parallel Programming Model}

Distributed Memory / Message Passing Model

\end{frame}

%------------------------------------------------------------
%------------------------------------------------------------

\section{Next steps}

%------------------------------------------------------------

\begin{frame}
\frametitle{Next steps}

Comments and suggestions?

\end{frame}

%------------------------------------------------------------
%------------------------------------------------------------

\section{Backup slides}

%------------------------------------------------------------

\begin{frame}
\frametitle{Backup Slides}

A rough layering:

\begin{center}
    \begin{tabular}{|c|} 
    \hline
    User application (sorting SAM) \\
    \hline
    User level libraries, such as MPI \\
    \hline
    Lower level libraries which implements RDMA \\
    \hline
    Hardware that supports RDMA \\
    \hline
    \end{tabular}
\end{center}

I guess the major part of my work would be \textbf{using} the user level libraries to write the user applications, 
instead of \textbf{writing} the user level libraries. 
What do you think?

\end{frame}

\end{document}
